\documentclass{article}%
\usepackage[T1]{fontenc}%
\usepackage[utf8]{inputenc}%
\usepackage{lmodern}%
\usepackage{textcomp}%
\usepackage{lastpage}%
%
%
%
\begin{document}%
\normalsize%
Paragraph(\newline%
'style': <ParagraphStyle 'Normal'>\newline%
'bulletText': None\newline%
'text': u'<para> Is it true that \textbackslash{}\textbackslash{}(x\^{}n + y\^{}n = z\^{}n\textbackslash{}\textbackslash{}) if \textbackslash{}\textbackslash{}(x,y,z\textbackslash{}\textbackslash{}) and \textbackslash{}\textbackslash{}(n\textbackslash{}\textbackslash{}) are positive integers?. Explain. </para>'\newline%
'encoding': 'utf8'\newline%
'caseSensitive': 1\newline%
'debug': 0\newline%
'frags': {[}ParaFrag(\_\_tag\_\_='para', bold=0, fontName='Helvetica', fontSize=10, greek=0, italic=0, link=None, rise=0, strike=0, sub=0, sup=0, text=u' Is it true that \textbackslash{}\textbackslash{}(x\^{}n + y\^{}n = z\^{}n\textbackslash{}\textbackslash{}) if \textbackslash{}\textbackslash{}(x,y,z\textbackslash{}\textbackslash{}) and \textbackslash{}\textbackslash{}(n\textbackslash{}\textbackslash{}) are positive integers?. Explain. ', textColor=Color(0,0,0,1), underline=0){]}\newline%
) \#Paragraph%
Paragraph(\newline%
'style': <ParagraphStyle 'Normal'>\newline%
'bulletText': None\newline%
'text': u'<para> Is it true that \textbackslash{}\textbackslash{}(x\^{}n + y\^{}n = z\^{}n\textbackslash{}\textbackslash{}) if \textbackslash{}\textbackslash{}(x,y,z\textbackslash{}\textbackslash{}) and \textbackslash{}\textbackslash{}(n\textbackslash{}\textbackslash{}) are positive integers?. Explain. </para>'\newline%
'encoding': 'utf8'\newline%
'caseSensitive': 1\newline%
'debug': 0\newline%
'frags': {[}ParaFrag(\_\_tag\_\_='para', bold=0, fontName='Helvetica', fontSize=10, greek=0, italic=0, link=None, rise=0, strike=0, sub=0, sup=0, text=u' Is it true that \textbackslash{}\textbackslash{}(x\^{}n + y\^{}n = z\^{}n\textbackslash{}\textbackslash{}) if \textbackslash{}\textbackslash{}(x,y,z\textbackslash{}\textbackslash{}) and \textbackslash{}\textbackslash{}(n\textbackslash{}\textbackslash{}) are positive integers?. Explain. ', textColor=Color(0,0,0,1), underline=0){]}\newline%
) \#Paragraph%
Paragraph(\newline%
'style': <ParagraphStyle 'Normal'>\newline%
'bulletText': None\newline%
'text': u'<para> To launch Android Studio, navigate to the /opt/android{-}studio/bin directory in a terminal and execute ./studio.sh. Or use a desktop file, see below. You may want to add /opt/android{-}studio/bin to your PATH environmental variable so that you can start Android Studio from any directory. </para>'\newline%
'encoding': 'utf8'\newline%
'caseSensitive': 1\newline%
'debug': 0\newline%
'frags': {[}ParaFrag(\_\_tag\_\_='para', bold=0, fontName='Helvetica', fontSize=10, greek=0, italic=0, link=None, rise=0, strike=0, sub=0, sup=0, text=u' To launch Android Studio, navigate to the /opt/android{-}studio/bin directory in a terminal and execute ./studio.sh. Or use a desktop file, see below. You may want to add /opt/android{-}studio/bin to your PATH environmental variable so that you can start Android Studio from any directory. ', textColor=Color(0,0,0,1), underline=0){]}\newline%
) \#Paragraph%
Paragraph(\newline%
'style': <ParagraphStyle 'Normal'>\newline%
'bulletText': None\newline%
'text': u'<para> Ionic apps are made of high{-}level building blocks called components. Components allow you to quickly construct an interface for your app. Ionic comes with a number of components, including modals, popups, and cards. Check out the examples below to see what each component looks like and to learn how to use each one. Once you\textbackslash{}u2019re familiar with the basics, head over to the API docs for ideas on how to customize each component. </para>'\newline%
'encoding': 'utf8'\newline%
'caseSensitive': 1\newline%
'debug': 0\newline%
'frags': {[}ParaFrag(\_\_tag\_\_='para', bold=0, fontName='Helvetica', fontSize=10, greek=0, italic=0, link=None, rise=0, strike=0, sub=0, sup=0, text=u' Ionic apps are made of high{-}level building blocks called components. Components allow you to quickly construct an interface for your app. Ionic comes with a number of components, including modals, popups, and cards. Check out the examples below to see what each component looks like and to learn how to use each one. Once you\textbackslash{}u2019re familiar with the basics, head over to the API docs for ideas on how to customize each component. ', textColor=Color(0,0,0,1), underline=0){]}\newline%
) \#Paragraph%
Paragraph(\newline%
'style': <ParagraphStyle 'Normal'>\newline%
'bulletText': None\newline%
'text': u'<para> Is \textbackslash{}\textbackslash{}((a+b)\^{}2=a\^{}2+b\^{}2+2ab\textbackslash{}\textbackslash{}) ? </para>'\newline%
'encoding': 'utf8'\newline%
'caseSensitive': 1\newline%
'debug': 0\newline%
'frags': {[}ParaFrag(\_\_tag\_\_='para', bold=0, fontName='Helvetica', fontSize=10, greek=0, italic=0, link=None, rise=0, strike=0, sub=0, sup=0, text=u' Is \textbackslash{}\textbackslash{}((a+b)\^{}2=a\^{}2+b\^{}2+2ab\textbackslash{}\textbackslash{}) ? ', textColor=Color(0,0,0,1), underline=0){]}\newline%
) \#Paragraph%
\end{document}